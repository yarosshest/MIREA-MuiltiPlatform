\section*{\LARGE Введение}
\addcontentsline{toc}{section}{Введение}

Flutter — комплект средств разработки и фреймворк с открытым исходным кодом для создания мобильных приложений под Android и iOS,
веб-приложений, а также настольных приложений под Windows, macOS и Linux с использованием языка программирования Dart,
разработанный и развиваемый корпорацией Google.

\textbf{Цель работы} --- познакомится с комплектом средств разработки и фреймворк Flutter.
План практической работы:
\begin{itemize}
	\item Установка фреймворка Flutter
	и окружения для языка программирования Dart;
	\item Установка интегрированной среды разработки Android Studio;
	\item Установка браузера Google Chrome;
	\item Установка и настройка требуемых расширений для Android Studio;
	\item Проверка корректности настройки системы;
	\item Создание проекта
	\item Запуск его на трех платформах:
	Android (эмулятор или реальное устройство), Windows и Web.
\end{itemize}

\clearpage

\section*{\LARGE Выполнение практической работы}
\addcontentsline{toc}{section}{Выполнение практической работы}

\section{Установка фреймворка Flutter и окружения для языка программирования Dart}

Для корректной установки SDK Flutter требуется перейти на официальный сайт Flutter и
перейти по ссылке Get Started.

После требуется выбрать операционную систему рабочей машины и перейти к инструкции
по установке на нее.
После выбора операционной системы требуется выбрать платформу для старта.
Выбрать нужно мобильную платформу

После перехода на страницу с инструкциями по установке Flutter SDK требуется произвести
указанные действия и установить как Flutter SDK на рабочую машину, так и переменную окружения
с путем до установленной Flutter SDK.

Для корректной установки SDK Flutter требуется перейти
на официальный сайт Flutter (Рисунок.~\ref{fig:FlutterOFF}).

\begin{image}
	\includegrph{img1.png}
	\caption{Официальный сайт Flutter}
	\label{fig:FlutterOFF}
\end{image}


\clearpage

После требуется выбрать операционную систему рабочей машины
и перейти к инструкции по установке на нее (Рисунок.~\ref{fig:getStarted}).

\begin{image}
	\includegrph{img2.png}
	\caption{get started}
	\label{fig:getStarted}
\end{image}


Выбрали платформу соответственно устройству, Windows (Рисунок.~\ref{fig:windowsFlutter}).

\begin{image}
	\includegrph{img3.png}
	\caption{windows}
	\label{fig:windowsFlutter}
\end{image}

\clearpage

После выбора операционной системы требуется выбрать платформу для старта.
Выбрали мобильною платформу (Рисунок.~\ref{fig:androidFlutterPage}).
Скачаем последнюю версию Flutter 3.24.3.

\begin{image}
	\includegrph{img4.png}
	\caption{Инструкция по установке Flutter}
	\label{fig:androidFlutterPage}
\end{image}


После установки нужно извлечь файл в каталог,
в котором мы хотим сохранить Flutter SDK (Рисунок~\ref{fig:install:unpack}).

\begin{image}
	\includegrph{flutter-unzip}
	\caption{Установка Flutter SDK в каталог}
	\label{fig:install:unpack}
\end{image}

\clearpage

Чтобы запускать команды Flutter в терминале,
нужно добавить Flutter в переменную среды PATH
(Рисунок~\ref{fig:install:path}).

\begin{image}
	\includegrph{flutter-path}
	\caption{Добавление путь до Flatter SDK в переменную среды}
	\label{fig:install:path}
\end{image}


\clearpage

\section{Установка интегрированной среды разработки Android Studio}

В начале инструкций по установке Flutter SDK находится перечень предлагаемых
для взаимодействия интегрированных сред разработки.\par
Для выполнения практических работ предлагается использовать Android Studio,
так как она поваляет проще взаимодействовать с Android устройствами.\par
Для скачивания Android Studio воспользуемся официальным сайтом.
Установим новейшую версию Android Studio (Рисунок~\ref{fig:and-stud-site}).
Скачанная версия ровна 2024.1.2


\begin{image}
	\includegrph{and-stud-site}
	\caption{Скачивание Android Studio}
	\label{fig:and-stud-site}
\end{image}

\clearpage

Запустим мастер установки Android Studio (Рисунок~\ref{fig:and-install}) и будем следовать шагам установки.

\begin{image}
	\includegrph[scale=0.4]{and-install}
	\caption{Установка Android Studio}
	\label{fig:and-install}
\end{image}



При выборе компонентов (Рисунок~\ref{fig:as:comp}) добавим AVD это свой эмулятор  Android Studio.

\begin{image}
	\includegrph[scale=0.5]{as-comp}
	\caption{Выбор компонентов}
	\label{fig:as:comp}
\end{image}
\clearpage

Выберем место установки Android Studio (Рисунок~\ref{fig:as:dist}).

\begin{image}
	\includegrph[scale=0.5]{as-dist}
	\caption{Выбор места установки}
	\label{fig:as:dist}
\end{image}


Создадим ярлык с Android Studio (Рисунок~\ref{fig:as:folder}).

\begin{image}
	\includegrph[scale=0.5]{as-folder}
	\caption{Создание ярдыка}
	\label{fig:as:folder}
\end{image}

\clearpage

Ждем установки (Рисунок~\ref{fig:as:wait}).

\begin{image}
	\includegrph[scale=0.5]{as-wait}
	\caption{Ждем установки}
	\label{fig:as:wait}
\end{image}

По окончанию запустим Android Studio (Рисунок~\ref{fig:as:incompl}).

\begin{image}
	\includegrph[scale=0.5]{as-incompl}
	\caption{Запуск Android Studio}
	\label{fig:as:incompl}
\end{image}


\clearpage

Сконфигурируем IDE Android Studio вместе с используемым эмулятором.
При первом открытии Android Studio нас встречает экран приветствия Android Studio Wizard (Рисунок~\ref{fig:as:welcom}).

\begin{image}
	\includegrph[scale=0.5]{welcom-as}
	\caption{Экран приветствия Android Studio Setup Wizard}
	\label{fig:as:welcom}
\end{image}

Пройдя экран приветствия, мы увидим окно выбора типа установки (Рисунок~\ref{fig:as:inn:type}), выберем Custom.

\begin{image}
	\includegrph[scale=0.5]{as-inn-type}
	\caption{Выбор Install Type}
	\label{fig:as:inn:type}
\end{image}

При выборе Custom Install Type нам будет предложено выбрать нужные компоненты для установки
(Рисунок~\ref{fig:as:zdk:in}), выберем их все и перейдем на следующий экран.

\begin{image}
	\includegrph[scale=0.5]{as-zdk-in}
	\caption{Предлагаемые для установки компоненты}
	\label{fig:as:zdk:in}
\end{image}

Следующим экраном является лицензионное соглашение (Рисунок~\ref{fig:as:zdk:in}), принимаем все условия.

\begin{image}
	\includegrph[scale=0.45]{as-lcz-agg}
	\caption{Экран лицензионного соглашения}
	\label{fig:as:lcz:agg}
\end{image}

\clearpage

Теперь сконфигурируем Android эмулятор для работы.
Откроем Device Manager (Рисунок~\ref{fig:as:dm:op}) и создадим новый эмулятор.

\begin{image}
	\includegrph[scale=0.5]{as-dm-op}
	\caption{Device Manager}
	\label{fig:as:dm:op}
\end{image}

В качестве устройства выберем Google Pixel 8 (Рисунок~\ref{fig:as:dm:cr}), указав это название в поисковой строке.

\begin{image}
	\includegrph[scale=0.35]{as-dm-cr}
	\caption{Выбор устройства}
	\label{fig:as:dm:cr}
\end{image}

\clearpage

В качестве образа выберем UpsideDownCake (Рисунок~\ref{fig:as:dm:ob}), который подразумевает использование API 34.

\begin{image}
	\includegrph[scale=0.25]{as-dm-ob}
	\caption{Выбор образа}
	\label{fig:as:dm:ob}
\end{image}


После выбора 34 API мы перейдем на конечный экран (Рисунок~\ref{fig:as:dm:as}), с просмотром всех характеристик эмулятора.
Здесь также можно изменить некоторые параметры в виде названия или внутри Advanced Settings.

\begin{image}
	\includegrph[scale=0.25]{as-dm-as}
	\caption{Результирующий экран создания эмулятора}
	\label{fig:as:dm:as}
\end{image}



\clearpage

\section{Установка браузера Google Chrome}

Для корректной работы с Web платформой потребуется установленный
на рабочей машине один из общедоступных браузеров.
В связи с тем, что Flutter был разработан компанией Google,
то SDK предлагает установить Google Chrome
для корректной работы с Web платформой.
Google Chrome версии 128.0.6613.138 был уже установлен на пк на котором выполняется практическая работа(Рисунок~\ref{fig:install:chrome}).

\begin{image}
	\includegrph{chrome-install}
	\caption{Установка Google Chrome}
	\label{fig:install:chrome}
\end{image}

\clearpage

\section{Установка и настройка требуемых расширений для Android Studio}

Когда среда разработки будет успешно установлена требуется
установить расширение для корректной работы с Flutter SDK
(Рисунок~\ref{fig:android:plugin}).

\begin{image}
	\includegrph{android-plugin}
	\caption{Flutter плагин для Android}
	\label{fig:android:plugin}
\end{image}


\clearpage

Так же после завершения всех настроек Android Studio,
требуется пройти процедуру, связанную с соглашением на использование
Android SDK командой \verb|flutter doctor --android-licenses|
(Рисунок~\ref{fig:dockor:android:licenses}).

\begin{image}
	\includegrph{dockor-android-licenses}
	\caption{Лицензионное соглашение Android}
	\label{fig:dockor:android:licenses}
\end{image}

\clearpage

\section{Проверка корректности настройки системы}

По окончанию всех установок и настроек рабочего окружения требуется
провести проверку готовности системы к работе.
Для этого в командной строке вызовем \texttt{flutter doctor -v}
(Рисунок~\ref{fig:flutter:dockor}).

\begin{image}
	\includegrph{flutter-dockor}
	\caption{Вызов flutter dockor}
	\label{fig:flutter:dockor}
\end{image}

Если система говорит, что не знает команды flutter,
то это означает, что либо Flutter SDK установлен не верно,
либо Flutter не внесен в переменные окружения.
Корректная работа команды должна вывести инструментарий
для работы с Flutter SDK,
а также статус настройки этого инструмента в виде галочки
или восклицательного знака. Настройка может быть завершена,
если галочки стоят у пунктов: \textit{Flutter}, \textit{Windows},
\textit{Android Toolchain}, \textit{Android Studio}
(или другой среду разработки, если устанавливали другую)
и \textit{Chrome} (если производили установку).
Если настройка прошла корректно, то можно приступать к созданию проекта.

\clearpage

\section{Создание проекта}

Для создания проекта в Android Studio требуется запустить среду разработки
и в открывшемся окне выбрать <<New Flutter Project>>.
Если же до этого уже был открыт другой проект,
то требуется открыть пункт меню <<File>>, в нем <<New>>
и в нем <<New Flutter project>>.\par
После чего среда разработки может попросить указать маршрут до Flutter SDK
и после попросит заполнить данные о проекте
(Рисунок~\ref{fig:setting:project}).

\begin{image}
	\includegrph{setting-project}
	\caption{Настройка проекта}
	\label{fig:setting:project}
\end{image}

В данных о проекте находится следующая информация:

\begin{itemize}
	\item Название проекта;
	\item Месторасположение будущего проекта;
	\item Описание проекта;
	\item Тип проекта --- должно быть выбрано Application;
	\item Организация;
	\item Нативный язык Android платформы;
	\item Нативный язык IOS платформы;
	\item Создаваемые платформы --- обязательно должны быть выбраны:
	Android, Windows/Mac, Web
\end{itemize}

Когда все значения установлены, можно нажимать клавишу <<Create>>
и ожидать создание и индексации нового тестового проекта.
\clearpage
\section{Запуск проекта}

Когда проект создастся и проиндексируется,
среда разработки позволит запустить проект на выбранной платформе.
Для регулирования платформы в верхнем тулбаре можно нажать на левый дропдаун,
который продемонстрирует все подключенные устройства различных платформ.\par
Для запуска приложения необходимо нажать на <<Запуск>>, в виде зеленой клавши
<<Воспроизвести>>.
При запуске проекта в консоли отладки появится информация
по запуску и проект запуститься на выбранном устройстве платформы.

Запущенное тестовое приложение на платформе Windows (Рисунок~\ref{fig:run:win}).

\begin{image}
	\includegrph{run-win}
	\caption{Запуск приложения на Windows}
	\label{fig:run:win}
\end{image}

\clearpage

Запущенное тестовое Web приложение (Рисунок~\ref{fig:run:web}).

\begin{image}
	\includegrph{run-web}
	\caption{Запуск web приложения}
	\label{fig:run:web}
\end{image}

\clearpage

Запущенное тестовое android приложение (Рисунок~\ref{fig:run:android}).

\begin{image}
	\includegrph{run-android}
	\caption{Запуск android приложения}
	\label{fig:run:android}
\end{image}
\clearpage


\section*{ВЫВОД}
\addcontentsline{toc}{section}{ВЫВОД}

В ходе выполнения практической работы была проведена установка
и настройка необходимых компонентов для разработки приложений
с использованием фреймворка Flutter и языка программирования Dart.
Была произведена установка интегрированной среды разработки Android Studio,
а также браузера Google Chrome,
необходимого для тестирования веб-версии приложений.\par
В Android Studio были установлены и настроены требуемые расширения
для работы с Flutter и Dart.
После этого была проведена проверка корректности настройки системы
с использованием команды flutter doctor,
которая подтвердила успешную настройку всех компонентов.\par
Далее был создан новый проект на Flutter,
который был успешно запущен на трех различных платформах:

\begin{itemize}
	\item Android --- проект был запущен на эмуляторе Android,
	а также протестирован на реальном устройстве.
	\item Windows --- была скомпилирована
	и запущена десктопная версия приложения.
	\item Web --- приложение было развернуто
	и протестировано в браузере Google Chrome.
\end{itemize}

Таким образом, все этапы практической работы были выполнены, и проект успешно заработал на всех указанных платформах.
